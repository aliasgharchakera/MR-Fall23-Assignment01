\documentclass[answers]{exam}

\usepackage{amsmath}
\usepackage{amssymb}
\usepackage{geometry}
\usepackage{venndiagram}
\usepackage{graphics}

% Header and footer.
\pagestyle{headandfoot}
\runningheadrule
\runningfootrule
\runningheader{EE/CE 468/468 Mobile Robotics}{Homework 1}{Fall 2023}
\runningfooter{}{Page \thepage\ of \numpages}{}
\firstpageheader{}{}{}

\boxedpoints
\printanswers

\newcommand{\uvec}[1]{\boldsymbol{\hat{\textbf{#1}}}}
\newcommand\union\cup
\newcommand\inter\cap
\newcommand\ul\underline
\newcommand\ol\overline

\title{Homework 1\\ EE/CE 468/468 Mobile Robotics\\ Habib University -- Fall 2023}
\author{Ali Asghar Yousuf\\ Muhammad Azeem Haider}  % replace with your ID, e.g. oy02945
\date{\today}

\begin{document}
\maketitle

\begin{questions}
\question Problem 1
\begin{parts}
\part Part a
\begin{solution}
    abc
\end{solution}
\part Part b
\begin{solution}
    For $R^w_a$, we have to rotate the $y_w$ axis along the $z_w$ axis by $-90^{\circ}$ and the $x_w$ axis along the $y_w$ axis by $-90^{\circ}$.

    \begin{equation*}
        R^w_a = R^w_y(-90^{\circ})R^w_x(-90^{\circ})
    \end{equation*}

    \begin{equation*}
        R^w_a = \begin{bmatrix}
            cos(-90^{\circ}) & -sin(-90^{\circ}) & 0 \\
            sin(-90^{\circ}) & cos(-90^{\circ})  & 0 \\
            0                & 0                 & 1
        \end{bmatrix}
        \begin{bmatrix}
            cos(-90^{\circ})  & 0 & sin(-90^{\circ}) \\
            0                 & 1 & 0                \\
            -sin(-90^{\circ}) & 0 & cos(-90^{\circ})
        \end{bmatrix}
    \end{equation*}

    \begin{equation*}
        R^w_a = \begin{bmatrix}
            0  & 1 & 0 \\
            -1 & 0 & 0 \\
            0  & 0 & 1
        \end{bmatrix}
        \begin{bmatrix}
            0 & 0 & -1 \\
            0 & 1 & 0  \\
            1 & 0 & 0
        \end{bmatrix}
    \end{equation*}

    \begin{equation*}
        R^w_a = \begin{bmatrix}
            0 & 1 & 0 \\
            0 & 0 & 1 \\
            1 & 0 & 0
        \end{bmatrix}
    \end{equation*}

    For $R^w_b$, we have to rotate the world frame along the $x_w$ axis by
    $90^{\circ}$.

    \begin{equation*}
        R^w_b = R^w_x(90^{\circ})
    \end{equation*}

    \begin{equation*}
        R^w_b = \begin{bmatrix}
            1 & 0               & 0                \\
            0 & cos(90^{\circ}) & -sin(90^{\circ}) \\
            0 & sin(90^{\circ}) & cos(90^{\circ})
        \end{bmatrix}
    \end{equation*}

    \begin{equation*}
        R^w_b = \begin{bmatrix}
            1 & 0 & 0  \\
            0 & 0 & -1 \\
            0 & 1 & 0
        \end{bmatrix}
    \end{equation*}
\end{solution}
\part Part c
\begin{solution}
We can leverage the fact that the rotation matrix is orthogonal and
that the inverse of an orthogonal matrix is its transpose.

\begin{equation*}
    (R^w_b)^{-1} = (R^w_b)^T
\end{equation*}
\begin{equation*}
    (R^w_b)^{-1} = \begin{bmatrix}
        1 & 0  & 0 \\
        0 & 0  & 1 \\
        0 & -1 & 0
    \end{bmatrix}
\end{equation*}

In order to verify that $(R^w_b)^{-1} = (R^w_b)^T$, we can use the inverse to
transform a vector from the $b$ frame to the $w$ frame.

$x$-axis in $b$ frame is $[1, 0, 0]^T$.
\begin{equation*}
(R^w_b)^{-1} [1, 0, 0]^T = \begin{bmatrix}
    1 & 0  & 0 \\
    0 & 0  & 1 \\
    0 & -1 & 0
\end{bmatrix}
\begin{bmatrix}
    1 \\
    0 \\
    0
\end{bmatrix}
= \begin{bmatrix}
1 \\
0 \\
0
\end{bmatrix}
\end{equation*}

$y$-axis in $b$ frame is $[0, 0, 1]^T$.
\begin{equation*}
(R^w_b)^{-1} [0, 0, 1]^T = \begin{bmatrix}
    1 & 0  & 0 \\
    0 & 0  & 1 \\
    0 & -1 & 0
\end{bmatrix}
\begin{bmatrix}
    0 \\
    0 \\
    1
\end{bmatrix}
= \begin{bmatrix}
0 \\
1 \\
0
\end{bmatrix}
\end{equation*}

$z$-axis in $b$ frame is $[0, -1, 0]^T$.
\begin{equation*}
(R^w_b)^{-1} [0, -1, 0]^T = \begin{bmatrix}
    1 & 0  & 0 \\
    0 & 0  & 1 \\
    0 & -1 & 0
\end{bmatrix}
\begin{bmatrix}
    0  \\
    -1 \\
    0
\end{bmatrix}
= \begin{bmatrix}
0 \\
0 \\
1
\end{bmatrix}
\end{equation*}

\end{solution}
\end{parts}
\question Problem 2
\begin{parts}
    \part Part a
    \begin{solution}
        abc
    \end{solution}
\end{parts}
\question Problem 3
\begin{parts}
    \part Part a
    \begin{solution}
        abc
    \end{solution}
\end{parts}
\question Problem 4
\begin{parts}
    \part Part a
    \begin{solution}
        abc
    \end{solution}
\end{parts}
\question Problem 5
\begin{parts}
    \part Part a
    \begin{solution}
        abc
    \end{solution}
\end{parts}

\end{questions}
\end{document}