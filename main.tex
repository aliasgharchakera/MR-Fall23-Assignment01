\documentclass[answers]{exam}

\usepackage{amsmath}
\usepackage{amssymb}
\usepackage{geometry}
\usepackage{venndiagram}
\usepackage{graphics}
\usepackage{siunitx}

% Header and footer.
\pagestyle{headandfoot}
\runningheadrule
\runningfootrule
\runningheader{EE/CE 468/468 Mobile Robotics}{Homework 1}{Fall 2023}
\runningfooter{}{Page \thepage\ of \numpages}{}
\firstpageheader{}{}{}

\boxedpoints
\printanswers

\newcommand{\uvec}[1]{\boldsymbol{\hat{\textbf{#1}}}}
\newcommand\union\cup
\newcommand\inter\cap
\newcommand\ul\underline
\newcommand\ol\overline

\title{Homework 1\\ EE/CE 468/468 Mobile Robotics\\ Habib University -- Fall 2023}
\author{Ali Asghar Yousuf\\ Muhammad Azeem Haider}  % replace with your ID, e.g. oy02945
\date{\today}

\begin{document}
\maketitle

\begin{questions}
    \question Problem 1
    \begin{parts}
        \part Part a
        \begin{solution}
            abc
        \end{solution}
        \part Part b
        \begin{solution}
            For $R^w_a$, we have to rotate the $y_w$ axis along the $z_w$ axis by $-90^{\circ}$ and the $x_w$ axis along the $y_w$ axis by $-90^{\circ}$.

            \begin{equation*}
                R^w_a = R^w_y(-90^{\circ})R^w_x(-90^{\circ})
            \end{equation*}

            \begin{equation*}
                R^w_a = \begin{bmatrix}
                    cos(-90^{\circ}) & -sin(-90^{\circ}) & 0 \\
                    sin(-90^{\circ}) & cos(-90^{\circ})  & 0 \\
                    0                & 0                 & 1
                \end{bmatrix}
                \begin{bmatrix}
                    cos(-90^{\circ})  & 0 & sin(-90^{\circ}) \\
                    0                 & 1 & 0                \\
                    -sin(-90^{\circ}) & 0 & cos(-90^{\circ})
                \end{bmatrix}
            \end{equation*}

            \begin{equation*}
                R^w_a = \begin{bmatrix}
                    0  & 1 & 0 \\
                    -1 & 0 & 0 \\
                    0  & 0 & 1
                \end{bmatrix}
                \begin{bmatrix}
                    0 & 0 & -1 \\
                    0 & 1 & 0  \\
                    1 & 0 & 0
                \end{bmatrix}
            \end{equation*}

            \begin{equation*}
                R^w_a = \begin{bmatrix}
                    0 & 1 & 0 \\
                    0 & 0 & 1 \\
                    1 & 0 & 0
                \end{bmatrix}
            \end{equation*}

            For $R^w_b$, we have to rotate the world frame along the $x_w$ axis by
            $90^{\circ}$.

            \begin{equation*}
                R^w_b = R^w_x(90^{\circ})
            \end{equation*}

            \begin{equation*}
                R^w_b = \begin{bmatrix}
                    1 & 0               & 0                \\
                    0 & cos(90^{\circ}) & -sin(90^{\circ}) \\
                    0 & sin(90^{\circ}) & cos(90^{\circ})
                \end{bmatrix}
            \end{equation*}

            \begin{equation*}
                R^w_b = \begin{bmatrix}
                    1 & 0 & 0  \\
                    0 & 0 & -1 \\
                    0 & 1 & 0
                \end{bmatrix}
            \end{equation*}
        \end{solution}
        \part Part c
        \begin{solution}
            We can leverage the fact that the rotation matrix is orthogonal and
            that the inverse of an orthogonal matrix is its transpose.

            \begin{equation*}
                (R^w_b)^{-1} = (R^w_b)^T
            \end{equation*}
            \begin{equation*}
                (R^w_b)^{-1} = \begin{bmatrix}
                    1 & 0  & 0 \\
                    0 & 0  & 1 \\
                    0 & -1 & 0
                \end{bmatrix}
            \end{equation*}

            In order to verify that $(R^w_b)^{-1} = (R^w_b)^T$, we can use the inverse to
            transform a vector from the $b$ frame to the $w$ frame.

            $x$-axis in $b$ frame is $[1, 0, 0]^T$.
            \begin{equation*}
                (R^w_b)^{-1} [1, 0, 0]^T = \begin{bmatrix}
                    1 & 0  & 0 \\
                    0 & 0  & 1 \\
                    0 & -1 & 0
                \end{bmatrix}
                \begin{bmatrix}
                    1 \\
                    0 \\
                    0
                \end{bmatrix}
                = \begin{bmatrix}
                    1 \\
                    0 \\
                    0
                \end{bmatrix}
            \end{equation*}

            $y$-axis in $b$ frame is $[0, 0, 1]^T$.
            \begin{equation*}
                (R^w_b)^{-1} [0, 0, 1]^T = \begin{bmatrix}
                    1 & 0  & 0 \\
                    0 & 0  & 1 \\
                    0 & -1 & 0
                \end{bmatrix}
                \begin{bmatrix}
                    0 \\
                    0 \\
                    1
                \end{bmatrix}
                = \begin{bmatrix}
                    0 \\
                    1 \\
                    0
                \end{bmatrix}
            \end{equation*}

            $z$-axis in $b$ frame is $[0, -1, 0]^T$.
            \begin{equation*}
                (R^w_b)^{-1} [0, -1, 0]^T = \begin{bmatrix}
                    1 & 0  & 0 \\
                    0 & 0  & 1 \\
                    0 & -1 & 0
                \end{bmatrix}
                \begin{bmatrix}
                    0  \\
                    -1 \\
                    0
                \end{bmatrix}
                = \begin{bmatrix}
                    0 \\
                    0 \\
                    1
                \end{bmatrix}
            \end{equation*}

        \end{solution}
    \end{parts}
    \question Problem 2
    \begin{parts}
        \part Part a
        \begin{solution}

        \end{solution}
    \end{parts}

    \question Problem 3
    \begin{parts}
        \part Part a
        \begin{solution}

            The general equation of a steerable wheel is as followed:

            \begin{align*}
                v^w_c = v^w_v + ({\omega}^v_s \times r^s_c) + ({\omega}^w_v \times r^v_c)
            \end{align*}

            Since there is no wheel offset $r^s_c$ will be 0. The equation will be simplified as followed:

            \begin{align*}
                v^w_c = v^w_v + ({\omega}^w_v \times r^v_c)
            \end{align*}

            The distance from the wheel to the center of the robot will be considered as l. 

            For wheel 1 where $\alpha_1 = 0^\circ$

            \begin{equation*}
                \begin{bmatrix}
                    V_{1x}  \\
                    V_{1y} \\
                    V_{1z}
                \end{bmatrix}
                = \begin{bmatrix}
                    V_x \\
                    V_y \\
                    0
                \end{bmatrix} 
                + \left(\begin{bmatrix}
                    0 \\
                    0 \\
                    \omega
                \end{bmatrix} \times \begin{bmatrix}
                    \cos(0) \cdot l \\
                    \sin(0) \cdot l \\
                    0
                \end{bmatrix}\right)
            \end{equation*}

            \begin{equation*}
                \begin{bmatrix}
                    V_{1x}  \\
                    V_{1y} \\
                    V_{1z}
                \end{bmatrix}
                = \begin{bmatrix}
                    V_x \\
                    V_y \\
                    0
                \end{bmatrix} 
                + \begin{bmatrix}
                    0 \\
                    l \cdot \omega \\
                    0
                \end{bmatrix}
            \end{equation*}

            \begin{equation} \label{eq:1}
                \begin{bmatrix}
                    V_{1x}  \\
                    V_{1y} \\
                    V_{1z}
                \end{bmatrix}
                = \begin{bmatrix}
                    V_x \\
                    V_y + l \cdot \omega\\
                    0
                \end{bmatrix} 
            \end{equation}
            

            For wheel 2 where $\alpha_1 = 120^\circ$

            \begin{equation*}
                \begin{bmatrix}
                    V_{2x}  \\
                    V_{2y} \\
                    V_{2z}
                \end{bmatrix}
                = \begin{bmatrix}
                    V_x \\
                    V_y \\
                    0
                \end{bmatrix} 
                + \left(\begin{bmatrix}
                    0 \\
                    0 \\
                    \omega
                \end{bmatrix} \times \begin{bmatrix}
                    \cos(120) \cdot l \\
                    \sin(120) \cdot l \\
                    0
                \end{bmatrix}\right)
            \end{equation*}

            \begin{equation*}
                \begin{bmatrix}
                    V_{2x}  \\
                    V_{2y} \\
                    V_{2z}
                \end{bmatrix}
                = \begin{bmatrix}
                    V_x \\
                    V_y \\
                    0
                \end{bmatrix} 
                + \begin{bmatrix}
                    - \frac{\sqrt[2]{3}}{2} \cdot l \cdot \omega \\
                    - \frac{1}{2} \cdot l \cdot \omega \\
                    0
                \end{bmatrix}
            \end{equation*}

            \begin{equation} \label{eq:2}
                \begin{bmatrix}
                    V_{2x}  \\
                    V_{2y} \\
                    V_{2z}
                \end{bmatrix}
                = \begin{bmatrix}
                    V_x - \frac{\sqrt[2]{3}}{2} \cdot l \cdot \omega \\
                    V_y - \frac{1}{2} \cdot l \cdot \omega \\
                    0
                \end{bmatrix} 
            \end{equation}

            For wheel 3 where $\alpha_1 = - 120^\circ$

            \begin{equation*}
                \begin{bmatrix}
                    V_{3x}  \\
                    V_{3y} \\
                    V_{3z}
                \end{bmatrix}
                = \begin{bmatrix}
                    V_x \\
                    V_y \\
                    0
                \end{bmatrix} 
                + \left(\begin{bmatrix}
                    0 \\
                    0 \\
                    \omega
                \end{bmatrix} \times \begin{bmatrix}
                    \cos(- 120) \cdot l \\
                    \sin(- 120) \cdot l \\
                    0
                \end{bmatrix}\right)
            \end{equation*}

            \begin{equation*}
                \begin{bmatrix}
                    V_{3x}  \\
                    V_{3y} \\
                    V_{3z}
                \end{bmatrix}
                = \begin{bmatrix}
                    V_x \\
                    V_y \\
                    0
                \end{bmatrix} 
                + \begin{bmatrix}
                    \frac{\sqrt[2]{3}}{2} \cdot l \cdot \omega \\
                    - \frac{1}{2} \cdot l \cdot \omega \\
                    0
                \end{bmatrix}
            \end{equation*}

            \begin{equation} \label{eq:3}
                \begin{bmatrix}
                    V_{3x}  \\
                    V_{3y} \\
                    V_{3z}
                \end{bmatrix}
                = \begin{bmatrix}
                    V_x + \frac{\sqrt[2]{3}}{2} \cdot l \cdot \omega \\
                    V_y - \frac{1}{2} \cdot l \cdot \omega \\
                    0
                \end{bmatrix} 
            \end{equation}

            Equations $\eqref{eq:1}$, $\eqref{eq:2}$, and $\eqref{eq:3}$ represent the velocity of contact point of the three wheels in the world frame. 

        \end{solution}
    \end{parts}

    \question Problem 4
    \begin{parts}
        \part Part a
        \begin{solution}
            To determine the command that will make the robot move in a clockwise direction in a circle with a radius of 1 unit and return to its starting position, we first need to establish the robot's pose:

            \begin{align*}
                \text{Pose} & = \begin{bmatrix}
                                    1        \\
                                    1        \\
                                    90^\circ \\
                                \end{bmatrix}
            \end{align*}

            The robot's initial position is at coordinates (1, 1) in the world frame, with
            a heading or orientation \(\phi = 90^\circ\). The distance between the robot's
            wheels is 1 unit.

            We can calculate the right and left wheel velocities using the following
            equations:

            \[\text{Right Wheel Velocity:} \quad V_r = (R - \frac{L}{2}) \cdot \omega\]
            \[\text{Left Wheel Velocity:} \quad V_l = (R + \frac{L}{2}) \cdot \omega\]

            Where:
            \begin{itemize}
                \item \(L\) is the distance between the centers of the two wheels.
                \item \(R\) is the signed distance from the Instantaneous Center of Curvature (ICC) to the midpoint between the wheels.
                \item \(\omega\) is the rate of rotation.
            \end{itemize}

            Given that the radius of the circle, \(R\), is 1 unit, we can calculate the
            wheel velocities as follows:

            \begin{align*}
                V_r & = (1 - \frac{1}{2}) \cdot \omega = 0.5 \cdot \omega \\
                V_l & = (1 + \frac{1}{2}) \cdot \omega = 1.5 \cdot \omega
            \end{align*}

            Therefore, the command to make the robot move in this circular path can be
            represented as:

            \[(V_l, V_r, t) = (1.5\cdot\omega, 0.5\cdot\omega, t)\]

            Now, assuming that we want the robot to complete exactly one full clockwise
            rotation before returning to its starting position, we can calculate the time
            duration (\(t\)) required. To do this, we need to determine the distance the
            robot needs to travel, which is equal to the circumference of the circle with a
            radius of 1 unit, \(2\pi\) units.

            Next, we calculate the linear velocity (\(V\)) of the robot as the average of
            the left and right wheel velocities:

            \[V = \frac{V_l + V_r}{2} = \frac{0.5\omega + 1.5\omega}{2} = \frac{2\omega}{2} = \omega\]

            So, for the given scenario where the robot needs to make one full circle with a
            radius of 1 unit and return to its starting position, you can set \(\omega\) to
            \(2\pi\).

            Now, we can calculate the time duration (t) using the formula:

            \[t = \frac{\text{Distance}}{V} = \frac{2\pi}{\omega}\]

            So, to make one full circle and return to the starting position, you can send
            the following command to the robot:

            \[(V_l, V_r, t) = (1.5\cdot\omega, 0.5\cdot\omega, \frac{2\pi}{\omega})\]

            Assuming \(\omega\) to be 1 radian, we can calculate the values of \((V_l, V_r,
            t)\) as follows:

            \begin{align*}
                V_l & = 1.5 \cdot \omega = 1.5 \cdot 1 \, \text{rad/s} = 1.5 \, \text{rad/s}    \\
                V_r & = 0.5 \cdot \omega = 0.5 \cdot 1 \, \text{rad/s} = 0.5 \, \text{rad/s}    \\
                t   & = \frac{2\pi}{\omega} = \frac{2\pi}{1 \, \text{rad/s}} = 2\pi \, \text{s}
            \end{align*}

            So, assuming \(\omega\) to be 1 radian, we have:

            \[
                (V_l, V_r, t) = (1.5 \, \text{rad/s}, 0.5 \, \text{rad/s}, 2\pi \, \text{s})
            \]
        \end{solution}
    \end{parts}
    \question Problem 5
    \begin{parts}
        \part Part a
        \begin{solution}
            abc
        \end{solution}
    \end{parts}

\end{questions}
\end{document}