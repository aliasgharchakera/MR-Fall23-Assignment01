\documentclass[answers]{exam}

\usepackage{amsmath}
\usepackage{amssymb}
\usepackage{geometry}
\usepackage{venndiagram}
\usepackage{graphics}
\usepackage{siunitx}

% Header and footer.
\pagestyle{headandfoot}
\runningheadrule
\runningfootrule
\runningheader{EE/CE 468/468 Mobile Robotics}{Homework 1}{Fall 2023}
\runningfooter{}{Page \thepage\ of \numpages}{}
\firstpageheader{}{}{}

\boxedpoints
\printanswers

\newcommand{\uvec}[1]{\boldsymbol{\hat{\textbf{#1}}}}
\newcommand\union\cup
\newcommand\inter\cap
\newcommand\ul\underline
\newcommand\ol\overline

\title{Homework 1\\ EE/CE 468/468 Mobile Robotics\\ Habib University -- Fall 2023}
\author{Ali Asghar Yousuf\\ Muhammad Azeem Haider}  % replace with your ID, e.g. oy02945
\date{\today}

\begin{document}
\maketitle

\begin{questions}
    \question Problem 1
    \begin{parts}
        \part Part a
        \begin{solution}
            here
        \end{solution}

        \part Part b
        \begin{solution}
            The rotation matrix of $R^w_a$ rotates from the world frame to frame $\{a\}$ and $R^w_b$ rotates from the world frame to frame $\{b\}$.
            
            There exists two rotations for $R^w_a$. 
            
            Using the standard rotational matrices, we get;

            $R^w_a$ = 


        
        \end{solution}
    \end{parts}
    \question Problem 2
    \begin{parts}
        \part Part a
        \begin{solution}
            
        \end{solution}
    \end{parts}
    \question Problem 3
    \begin{parts}
        \part Part a
        \begin{solution}
            abc
        \end{solution}
    \end{parts}
    \question Problem 4
    \begin{parts}
        \part Part a
        \begin{solution}
            We first need the pose of the robot to get the robot to move in a clockwise position from its initial position and bring it back to its starting position. The pose of the Alphabot is as follows:

            \begin{align*}
                \text{Pose} &= \begin{bmatrix}
                                    1 \\
                                    1 \\
                                    90^\circ \\
                                \end{bmatrix}
            \end{align*}

            We want the robot to move in a radius of 1 unit, and the distance between the width of the robot wheels is 1 unit.

            The equations for the right and left wheel velocities are given by:

            \[\text{Right Wheel Velocity:} \quad V_r = (R - \frac{L}{2}) \cdot \omega\]
            \[\text{Left Wheel Velocity:} \quad V_l = (R + \frac{L}{2}) \cdot \omega\]

            Where:
            \begin{itemize}
                \item $L$ is the distance between the centers of the two wheels.
                \item $R$ is the signed distance from the Instantaneous Center of Curvature (ICC) to the midpoint between the wheels.
                \item $\omega$ is the rate of rotation.
            \end{itemize}

            In this case, the radius of the circle, $R$, is 1 unit.

            \begin{align*}
                V_r &= (1 - \frac{1}{2}) \cdot \omega = 0.5 \cdot \omega \\
                V_l &= (1 + \frac{1}{2}) \cdot \omega = 1.5 \cdot \omega
            \end{align*}

            \[(V_l, V_r, t) = (1.5\cdot\omega, 0.5\cdot\omega, t)\]

            Assuming that we want the robot to make only one full clockwise rotation before getting back to the starting position, we can calculate the time duration (t) as follows:

            First, we need to calculate the circumference of the circle with a radius of 1 unit, which is \(2\pi\) units. The robot will need to travel this distance.

            Next, we can calculate the linear velocity of the robot, which is the average of the left and right wheel velocities:

            \[V = \frac{V_l + V_r}{2}\]

            We can find V as:

            \[V = \frac{0.5\omega + 1.5\omega}{2} = \frac{2\omega}{2} = \omega\]

            Now, we can calculate the time duration (t) using the formula:

            \[t = \frac{\text{Distance}}{V} = \frac{2\pi}{\omega}\]

            So, to make one full circle and return to the starting position, you can send the following command to the robot:

            \[(V_l, V_r, t) = (1.5\cdot\omega, 0.5\cdot\omega, \frac{2\pi}{\omega})\]

        \end{solution}
    \end{parts}
    \question Problem 5
    \begin{parts}
        \part Part a
        \begin{solution}
            abc
        \end{solution}
    \end{parts}

\end{questions}
\end{document}